% !TeX root = ./mfog-ieee.tex
% \documentclass{article}
% \documentclass[conference]{lib/IEEEtran}
\documentclass[conference]{IEEEtran}
% \documentclass[runningheads]{llncs}
\usepackage{babel}
\usepackage[utf8]{inputenc}
\IEEEoverridecommandlockouts
% The preceding line is only needed to identify funding in the first footnote. If that is unneeded, please comment it out.
\usepackage{xspace}
% \usepackage{cite}
\usepackage{amsmath,amssymb,amsfonts}
\usepackage{algorithmic}
\usepackage{graphicx}
\usepackage{textcomp}
\usepackage{xcolor}
% \def\BibTeX{{\rm B\kern-.05em{\sc i\kern-.025em b}\kern-.08em
%     T\kern-.1667em\lower.7ex\hbox{E}\kern-.125emX}}
\usepackage{hyperref}
% \usepackage[clean]{svg}
\usepackage{subcaption}
\usepackage{mdframed}% http://ctan.org/pkg/mdframed

\usepackage[colorinlistoftodos,prependcaption,textsize=tiny]{todonotes}
\newcounter{todocounter}

\hypersetup{colorlinks=true, allcolors=black}
\begin{document}

\title{IoT Data Stream Novelty Detection:
Design, Implementation and Evaluation [DRAFT]\thanks{CNPq}}

\author{
  \IEEEauthorblockN{Luís Puhl, Guilherme Weigert Cassales, Hermes Senger, Helio Crestana Guardia}
  \IEEEauthorblockA{Universidade Federal de São Carlos, Brasil \\
    Email: \{luispuhl, gwcassales\}@gmail.com, \{hermes@, helio@dc.\}ufscar.br
  }
  % \IEEEauthorblockN{Hermes Senger}\IEEEauthorblockA{\textit{Universidade Federal de São Carlos}, Brasil \\hermes@ufscar.br}\and
    % \IEEEauthorblockN{Guilherme Weigert Cassales}\IEEEauthorblockA{\textit{Universidade Federal de São Carlos}, Brasil \\gwcassales@gmail.com}
    % \and
    % \IEEEauthorblockN{4\textsuperscript{th} Given Name Surname}
    % \IEEEauthorblockA{\textit{dept. name of organization (of Aff.)} \\
    % \textit{name of organization (of Aff.)}\\
    % City, Country \\
    % email address or ORCID}
}

\maketitle

\newcommand{\toreview}{}
\newcommand{\reffig}[1]{Figure \ref{fig:#1}\xspace}

\ifx\toreview\undefined
  \newcommandx{\nota}[2][1=]{}
  \newcommand{\hl}[1]{}

  \newcommandx{\notahl}[2][1=]{}
  \newcommand{\hlhl}[1]{}
  \newcommandx{\notake}[2][1=]{}
  \newcommand{\hlke}[1]{}
  \newcommandx{\notafa}[2][1=]{}
  \newcommand{\hlfa}[1]{}
\else
  % \newcommandx{\nota}[2][1=]{
  %   \stepcounter{todocounter}
  %   \todo[linecolor=red,backgroundcolor=red!25,bordercolor=red,#1]{[\thetodocounter] #2}}
  \newcommand{\hl}[1]{\colorbox{red!25}{#1}}
  % 
  \newenvironment{highlight}{\begin{mdframed}[backgroundcolor=red!25]}{\end{mdframed}}
  % \newenvironment{myHeartEnv}
  % {\color{purple}{\heartsuit}\kern-2.5pt\color{green}{\heartsuit}}
  % {\text{ forever}}

  % \newcommandx{\notahl}[2][1=]{\stepcounter{todocounter}\todo[
  %   linecolor=green,
  %   backgroundcolor=green!25,
  %   bordercolor=green,#1]{[Helio \thetodocounter] #2}}
  % \newcommand{\hlhl}[1]{\colorbox{green}{#1}}

  % \usepackage{cancel}
  % \newcommandx{\notake}[2][1=]{\stepcounter{todocounter}\todo[
  %   linecolor=yellow,
  %   backgroundcolor=yellow!25,
  %   bordercolor=yellow,#1]{[Kelton \thetodocounter] #2}}
  % \newcommand{\hlke}[1]{\colorbox{yellow}{#1}}

  % \newcommandx{\notafa}[2][1=]{\stepcounter{todocounter}\todo[
  %   linecolor=blue,
  %   backgroundcolor=blue!25,
  %   bordercolor=blue,#1]{[Faria \thetodocounter] #2}}
  % \newcommand{\hlfa}[1]{\colorbox{blue!30}{#1}}

  % \usepackage[showframe,layoutvoffset=0mm,includeall,tmargin=15mm,bmargin=20mm]{geometry}
  % \marginparwidth=34mm
  % \hoffset=-25mm
  % \textwidth=165mm
\fi

\begin{abstract}
  
\end{abstract}

\begin{IEEEkeywords}
% Detecção de Novidades, Detecção de Intrusão, Fluxos de Dados, Computação Distribuı́da, Computação em Névoa, Internet das Coisas.
novelty detection, intrusion detection, data streams, distributed system, edge computing, internet of things
\end{IEEEkeywords}


% ----------------------------------------------------------------------------------------------------------------------
% - Escrever! Corpo:
%   - Proposal (retomar problema {iot, sec, ND}, objetivo, soluções {minas, paralelismo, distribuído, ~~py-kafka, flink,~~ mpi}, propor uma solução)
%   - Implementation (mpi, c, data-structures, data-flow, )
%   - Experiments (rpi, cluster, `evaluate.py`)
%   - Results
%   - Conclusion
% - Demonstrar o paralelismo com figura de pipeline (time vs instruction)

% \bibliography{IEEEabrv,refs.bib}
\bibliographystyle{lib/IEEEtran.bst}
\bibliography{refs.bib}
\end{document}
