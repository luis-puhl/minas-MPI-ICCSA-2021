% !TeX root = ./00.main.ICCSA.tex

\section{MINAS}
\label{sec:minas}

%  ******************* Texto quali ************************
% \subsection{O Algoritmo MINAS}
% Apresentar o MINAS de forma resumida (1 pagina no max): como funciona, etc. 

\minas \cite{Faria2013Minas,Faria2016minas} is an offline-online \nd algorithm,
meaning it has two distinct phases. The first phase (offline) creates an initial
model set with several clusters based on a clustering algorithm with a labeled
training set.
Each cluster can be associated with only one class of the problem, but each
class can have many clusters.

During its online phase, which is the main focus of our work, \minas performs
three tasks in (near) real-time over a potentially infinite data stream:
% Our proposal focuses on the online phase, therefore we will describe it in more detail.
% We describe the online phase in more detail since we focus on its tasks.
% in summary,
%  the online phase of \minas executes 
classification, novelty detection and model update,
as shown in Algorithm \ref{alg:MINAS}.

\minas attempts to classify each incoming unlabeled instance according to the
current decision model. Instances not explained by the current model
receive the \textit{``unknown''} label and are stored in the unknowns-buffer.
%  "in the" vs "in a": Como só exite um buffer, único, achei que fosse THE.
When the unknowns-buffer size reaches a preset threshold, \minas executes the
% (parametric) 
Novelty Detection function.
% Não entendi o "beyond the ND..." - vejam se fez sentido - helio
After a set interval, samples in the unknowns-buffer are considered to be
noise or outliers and removed.
The algorithm also has a mechanism to forget clusters that became obsolete and
unrepresentative of the current data stream distribution, removing them from
the Model and storing in a Sleep Model for recurring patterns
detection \cite{Faria2016minas}.
% cluster in the model with no recent match
% are removed as well, as they correspond to obsolete and nonrepresentative
% patterns in the current data stream 
% Beyond the Novelty Detection, \minas also cleans the unknowns-buffer removing
% old instances as they represent noise or outliers and, has a 

\begin{algorithm}[htb]
    % \DontPrintSemicolon
    \SetKwFunction{nearestCluster}{nearestCluster}
    \SetKwFunction{clustering}{clustering}
    \SetKwFunction{NoveltyDetection}{NoveltyDetection}
    \SetKwFunction{handleModelSleep}{moveToSleep}
    \SetKwFunction{removeOldSamples}{removeOldSamples}
    % 
    \SetKwProg{Function}{Function}{:}{}
    \SetKwFor{With}{with}{}{}
    \SetKw{continue}{continue}
    % 
    \KwIn{ModelSet, inputStream}
    \KwOut{outputStream}
    % 
    \SetKwData{cleaningWindow}{cleaningWindow}
    \SetKwData{noveltyDetectionTrigger}{noveltyDetectionTrigger}
    \SetKwInOut{KwParams}{Parameters}
    \KwParams{\cleaningWindow, \noveltyDetectionTrigger}
    % \KwSty{Parameters}: \cleaningWindow, \noveltyDetectionTrigger\\
    % 
    \SetKwFunction{MinasOnline}{MinasOnline}
    \Function{\MinasOnline{Model, inputStream}}{
        UnknownsBuffer $\leftarrow \emptyset$; SleepModel $\leftarrow \emptyset$ \;
        lastCleanup $\leftarrow 0$; noveltyIndex $\leftarrow 0$\;
        % sampleIn $\leftarrow 0$\;
        \ForEach{ {$sample_{i}$} $\in$ inputStream }{
            % sample.label $\leftarrow$ unknown\;
            % (distance, cluster) $\leftarrow$ \nearestCluster(sample, Model)\;
            nearest $\leftarrow$ \nearestCluster(sample, Model)\;
            \eIf{nearest.distance $\leq$ nearest.cluster.radius}{
                sample.label $\leftarrow$ nearest.cluster.label\;
                nearest.cluster.lastUsed $ \leftarrow i $ \;
            }
            {
                sample.label $\leftarrow$ unknown\;
                UnknownsBuffer $\leftarrow$ UnknownsBuffer $\cup$ sample\;
                \If{$|\;UnknownsBuffer\;| \geq$ \noveltyDetectionTrigger}{
                    % \tcc{Novelty Detection}
                    novelties $\leftarrow$ \NoveltyDetection(Model $\cup$ SleepModel, *UnknownsBuffer)\;
                    Model $\leftarrow$ Model $\cup$ novelties\;
                }
                \If{ $ i > $ ( lastCleanup $ + $ \cleaningWindow )}{
                    Model $\leftarrow$ \handleModelSleep(Model, *SleepModel, lastCleanup)\;
                    UnknownsBuffer $\leftarrow$ \removeOldSamples(UnknownsBuffer, lastCleanup)\;
                    lastCleanup $ \leftarrow i $\;
                }
            }
            outputStream.append(sample)\;
        }
    }
\caption{Our interpretation of \minas \cite{Faria2016minas}.}
\label{alg:MINAS}
\end{algorithm}

The Novelty Detection function, illustrated in Algorithm \ref{alg:MINAS-nd},
groups the instances to form new clusters, and each new cluster is validated to
discard the non-cohesive or unrepresentative ones.
Valid clusters are analyzed to decide if they represent an extension of a
known pattern or a completely new pattern. In both cases, the model absorbs the
valid clusters and starts using them to classify new instances.

\begin{algorithm}[htb]
    \SetKwProg{Function}{Function}{:}{}
    \SetKwData{minExamplesPerCluster}{minExamplesPerCluster}
    \SetKwData{noveltyFactor}{noveltyFactor}
    \SetKwInOut{KwParams}{Parameters}
    \KwParams{\minExamplesPerCluster, \noveltyFactor}
    % 
    \SetKwFunction{nearestCluster}{nearestCluster}
    \SetKwFunction{clustering}{clustering}
    \SetKwFunction{NoveltyDetection}{NoveltyDetection}
    \SetKwFunction{handleModelSleep}{moveToSleep}
    \SetKwFunction{removeOldSamples}{removeOldSamples}
    % 
    \Function{\NoveltyDetection{Model, Unknowns}}{
        newModelSet $\leftarrow$ $\emptyset$\;
        \ForEach{new in \clustering(Unknowns)}{
            \If{$(\,|new.sampleSet| \geq$ \minExamplesPerCluster$) \land (new.silhouette > 0)$}{
                nearest $\leftarrow$ \nearestCluster(new, Model)\;
                \eIf{nearest.distance $<$ (nearest.cluster.radius $\times$ \noveltyFactor)}{
                    new.label $\leftarrow$ nearest.cluster.label\;
                    new.type $\leftarrow$ ``extension''\;
                }{
                    new.label $\leftarrow$ noveltyIndex\;
                    noveltyIndex $\leftarrow$ noveltyIndex $+ 1$\;
                    new.type $\leftarrow$ ``novelty''\;
                }
                Unknowns $\leftarrow$ Unknowns $-$ new.sampleSet\;
                \label{alg:MINAS-nd:reclassify}
                newModelSet $\leftarrow$ newModelSet $\cup$ new\;
            }
        }
        \Return{newModelSet}\;
    }
    \caption{\minas \cite{Faria2016minas} Novelty Detection task.}
    \label{alg:MINAS-nd}
\end{algorithm}
