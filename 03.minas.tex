\section{MINAS {\color{red} Guilherme pode fazer}}
\label{sec:minas}


%  ******************* Texto quali ************************
% \subsection{O Algoritmo MINAS}
Apresentar o MINAS de forma resumida (1 pagina no max): como funciona, etc. 
% Veja se a Quali do Guilherme tem algum techo que possa inspirar 

O algoritmo da fase de aplicação da técnica MINAS é ilustrado no Algoritmo \ref{alg:MINAS}. Neste algoritmo estão inclusas as tarefas de classificar novos exemplos, detectar os padrões-novidade e atualizar o modelo de decisão. Este algoritmo recebe como argumentos o modelo construído na fase \textit{offline}, os dados do fluxo, um limiar $T$ utilizado para determinar se o exemplo é extensão ou novidade, o número mínimo de exemplos que um grupo necessita para ser válido, um limiar $ts$ utilizado na remoção de exemplos muito antigos da memória temporária e um limite de tempo $P$ utilizado no processo de mover padrões para memória \textit{sleep}.


\todo[inline, color=blue]{explicação do algoritmo em pseudocódigo}
As memórias temporária e \textit{sleep} são inicializadas com conjuntos vazios (linhas 1 e 2). Para cada novo exemplo não rotulado, o modelo de decisão verifica se é capaz de explicá-lo, ou seja, classificá-lo em alguma classe já existente. Neste processo, calcula-se a distância euclidiana $Dist$ entre o exemplo e o micro-cluster mais próximo (linhas 3 e 4). Se $Dist$ for menor que o raio do micro-cluster, este exemplo receberá o rótulo do micro-cluster e as estatísticas de resumo do micro-cluster serão atualizadas (linhas 5 a 7). Se nenhum dos micro-clusters existentes puder explicar o exemplo, ele será rotulado como $desconhecido$ e movido para uma memória temporária para análise futura (linhas 10 e 11). Então, há três significados possíveis para este exemplo: ($i$) ruído, ($ii$) evolução de conceitos conhecidos ou ($iii$) um novo conceito que ainda não está presente no modelo atual. Nos casos ($ii$) e ($iii$), um grupo coeso de exemplos é necessário para atualizar o modelo sem \textit{feedback}. Se o número de exemplos na memória temporária atingir um tamanho parametrizável, então a etapa de detecção da novidade é realizada (linhas 12 e 13). Finalmente, depois que cada janela de dados é processada, o modelo de decisão deve esquecer os micro-clusters que não são representativos, ou seja, que não receberam novos exemplos, movendo-os para uma memória de suspensão. Além disso exemplos desconhecidos que não foram adicionados a padrões novidade devem ser esquecidos por se tratarem de ruído (linhas 17 a 19).


% %\begin{scriptsize}
%     \begin{algorithm}[ht]
%     \caption{MINAS}
%     \label{alg:MINAS}
%     \renewcommand{\algorithmicrequire}{\textbf{Entrada:}}
%     \begin{algorithmic}[1]
%     %T = limiar de distância para pertencer ao grupo
%     %P = tempo de "inatividade" para passar para memória sleep
%     %ts = limiar para remoção de exemplos da memória temporária
%     \REQUIRE $Modelo,FCD,T,NumMinExemplos,ts,P$
%     \STATE $MemTmp \leftarrow \emptyset$
%     \STATE $MemSleep \leftarrow \emptyset$
%     \FORALL{$exemplo \in FCD$}
%     \STATE $(Dist,micro) \leftarrow$ micro-mais-proximo($exemplo,Modelo$)
%     \IF{$Dist < $ raio($micro$)}
%     \STATE $exemplo.classe \leftarrow micro.rotulo$
%     \STATE atualizar-micro($micro,exemplo$)
%     \ELSE
%     \STATE $exemplo.classe \leftarrow desconhecido$
%     \STATE $MemTmp \leftarrow MemTmp \cup exemplo$
%     \IF{$|MemTmp| \geq NumMinExemplos$}
%     \STATE $Modelo \leftarrow $ deteccao-novidade($Modelo,MemTmp,T$)
%     \ENDIF
%     \ENDIF
%     \STATE $TempoAtual \leftarrow exemplo.T$
%     \IF{$TempoAtual$ mod $TamJanela == 0$}
%     \STATE $Modelo \leftarrow$ mover-micro-grupos-mem-sleep($Modelo,MemSleep,P$)
%     \STATE $MemTmp \leftarrow$ remover-exemplos-antigos($MemTmp,ts$)
%     \ENDIF
%     \ENDFOR
%     \end{algorithmic}
%   \end{algorithm}
%  ******************* Texto quali ************************



Citar que o trabalho anterior já validou o uso do MINAs para detecção de novidade porem com uma implementação sequencial. 


Figura do MINAS (offline + online) ...

