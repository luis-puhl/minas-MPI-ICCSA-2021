
\begin{abstract}
  The ongoing implementation of the Internet of Things (IoT) is sharply increasing the small
  devices count and variety on edge networks and, following this increase, the
  attack opportunities for hostile agents also increases, requiring more from
  the network administrator role and the need for tools to detect and react to those
  threats.
  % 
  One such tool are the Network Intrusion Detection Systems (NIDS) where the network
  traffic is captured and analysed raising alarms when a known attack pattern or
  new pattern is detected.
  For a network security tool to operate in the context of edge and
  IoT it has to comply with processing time, storage space and energy
  requirements alongside traditional requirements for stream and network
  analysis like accuracy and scalability.
  % 
  This work addresses the construction details and evaluation of an prototype
  distributed IDS using MINAS Novelty Detection algorithm
  following up the previously defined IDSA-IoT architecture.
  We discuss the algorithm steps, how it can be deployed in a distributed environment,
  the impacts on the accuracy of MINAS and evaluate the performance and scalability
  using a cluster of constrained devices commonly found in IoT scenarios.
  % 
  We found an increase of \textit{A 0.0} processed network flow descriptors per core
  added to the cluster. Also \textit{B 0.0\%} and \textit{C 0.0\%} change in
  \textit{F1Score} in the tested datasets when stream was unlimited in speed and
  limited to \textit{0.0 z MB/s} respectively.
\end{abstract}

\ifdefined\IEEEkeywords
\begin{IEEEkeywords}
  % Detecção de Novidades, Detecção de Intrusão, Fluxos de Dados, Computação Distribuı́da, Computação em Névoa, Internet das Coisas.
  novelty detection, intrusion detection, data streams,
  distributed system, edge computing, internet of things
\end{IEEEkeywords}
\else
\keywords{
novelty detection \and intrusion detection \and data streams \and
distributed system \and edge computing \and internet of things
}
\fi
