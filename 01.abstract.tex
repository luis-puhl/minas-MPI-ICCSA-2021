\newcommand{\iot}{IoT\xspace}
\newcommand{\arch}{IDSA-IoT\xspace}
\newcommand{\nids}{NIDS\xspace}
\newcommand{\nd}{DSND\xspace}
\newcommand{\minas}{MINAS\xspace}
\newcommand{\refminas}{\textit{Ref}\xspace}
\newcommand{\mfog}{\textit{MFOG}\xspace}

\begin{abstract}
% cenário, problema, 
  The ongoing implementation of the Internet of Things (\iot) is sharply
  increasing the number and variety of small devices on edge networks.  
%   break to 2-3 lines
  Likewise, the attack opportunities for hostile agents also
  increases, requiring more effort from network administrators and strategies
  to detect and react to those threats.
  % 
  % One such tool are the Network Intrusion Detection Systems (\nids)
  % which captures and analyses network traffic,
  % acting when a known attack or a new pattern is detected.
  For a network security system to operate in the context of edge and
  \iot, it has to comply with processing, storage, and energy
  requirements alongside traditional requirements for stream and network
  analysis like accuracy and scalability.
  % 
  Using a previously defined architecture (\arch), we address the construction
  and evaluation of a support mechanism for distributed Network Intrusion
  Detection Systems (\nids) based on the \minas Data Stream Novelty Detection
  (\nd) algorithm.
  % MINAS employing MPI library.
  % of a distributed Novelty Detection System with the Data Stream Novelty
  % Detection (\nd) algorithm \minas and MPI library and we evaluate our proposal
  % in a Network Introduction Detection (\nids) role.
%   apresentar a estratégia,
  % 
  We discuss the algorithm steps, how it can be deployed in a distributed
  environment, the impacts on the accuracy and evaluate performance and
  scalability using a cluster of constrained devices commonly found in \iot
  scenarios.
  % 
  % We found an increase of \textit{A 0.0} processed network flow descriptors per
  % core added to the cluster.
  % Also \textit{B 0.0\%} and \textit{C 0.0\%} change in
  % \textit{F1Score} in the tested data sets when stream was unlimited in speed and
  % limited to \textit{0.0 z MB/s} respectively.
  % Results show a negligible loss of accuracy (hits) between original, serial and
  % distributed executions while using less time than previous implementation,
  % however efficient distribution was not achieved as the observed time as we
  % added nodes remained constant.
  % Segue proposta para o parágrafo acima - helio
  The obtained results show a negligible accuracy loss in the distributed
  version but also a small reduction in the execution time using low profile
  devices. Although not efficient, the parallel version showed to be viable as
  the proposed granularity provides equivalent accuracy and viable response times.
  % , consuming a months worth of flow descriptors in $300$ seconds.
  % We also found some per
%   concluindo que é viável e barato utiliar este tipo de abordagem
  % \todo[inline]{Manter IDS? Fazer novelty detection distribuído utilizando
  % recursos na borda, em pequenos dispositivos buscado escalabilidade (?) e baixa
  % latência, Os resultados mostram que ..., permitindo concluir que ... }
\end{abstract}

\ifdefined\IEEEkeywords
\begin{IEEEkeywords}
  % Detecção de Novidades, Detecção de Intrusão, Fluxos de Dados, Computação Distribuı́da, Computação em Névoa, Internet das Coisas.
  novelty detection, intrusion detection, data streams,
  distributed system, edge computing, internet of things
\end{IEEEkeywords}
\else
\keywords{
novelty detection \and intrusion detection \and data streams \and
distributed system \and edge computing \and internet of things
}
\fi
