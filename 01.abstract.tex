\newcommand{\refminas}{\textit{Ref}\xspace}
\newcommand{\mfog}{\textit{MFOG}\xspace}
\newcommand{\iot}{IoT\xspace}
\newcommand{\nids}{NIDS\xspace}
\newcommand{\nd}{DSND\xspace}
\newcommand{\minas}{MINAS\xspace}

\begin{abstract}

  The ongoing implementation of the Internet of Things (\iot) is sharply
  increasing the number and variety of small devices on edge networks and,
  following this increase, the attack opportunities for hostile agents also
  increases, requiring more from network administrators and the need for tools
  to detect and react to those threats.
  % 
  One such tool are the Network Intrusion Detection Systems (\nids)
  which captures and analyses network traffic,
  acting when a known attack or a new pattern is detected.
  For a network security tool to operate in the context of edge and
  \iot it has to comply with processing time, storage space and energy
  requirements alongside traditional requirements for stream and network
  analysis like accuracy and scalability.
  % 
  Using a previously defined architecture (IDSA-IoT),
  we address the construction and evaluation of a prototype distributed \nids
  with the Data Stream Novelty Detection algorithm (\nd) MINAS
  employing C and MPI library.
  % 
  We discuss the algorithm steps, how it can be deployed in a distributed
  environment, the impacts on the accuracy and evaluate performance and
  scalability using a cluster of constrained devices commonly found in \iot
  scenarios.
  % 
  % We found an increase of \textit{A 0.0} processed network flow descriptors per
  % core added to the cluster.
  % Also \textit{B 0.0\%} and \textit{C 0.0\%} change in
  % \textit{F1Score} in the tested datasets when stream was unlimited in speed and
  % limited to \textit{0.0 z MB/s} respectively.
  We found a marginal ($2$ to $4\%$) difference in true positive (hits) between
  original, serial and distributed executions, consuming a months worth of flow
  descriptors in $300$ seconds.
\end{abstract}

\ifdefined\IEEEkeywords
\begin{IEEEkeywords}
  % Detecção de Novidades, Detecção de Intrusão, Fluxos de Dados, Computação Distribuı́da, Computação em Névoa, Internet das Coisas.
  novelty detection, intrusion detection, data streams,
  distributed system, edge computing, internet of things
\end{IEEEkeywords}
\else
\keywords{
novelty detection \and intrusion detection \and data streams \and
distributed system \and edge computing \and internet of things
}
\fi
