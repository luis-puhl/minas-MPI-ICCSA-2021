\section{Introduction} 


\newcommand{\refminas}{\textit{Ref}\xspace}
\newcommand{\mfog}{\textit{MFOG}\xspace}
\newcommand{\iot}{IoT\xspace}
\newcommand{\nids}{NIDS\xspace}
\newcommand{\ds}{DS\xspace}

\todo[inline]{Atualizar o iscurso e tirar um pouco o foco de IDS para detecção de novidades na rede}

A survey released by Gartner in 2017 estimated that by 2020 there will be about
20 billion devices connected to the Internet, many of them through the
IoT\cite{gartner_forecast_2017}. Another survey released by the same company in
2018 estimated that nearly 20\% of organizations have experienced at least one
IoT-based attack in the last three years. The study shows that most
organizations have no control over the origin and nature of software and
hardware used by connected devices. To protect against these threats, IoT's
worldwide spending on security will increase from \$1.5 billion in 2018 to \$3.1
billion in 2021 \cite{gartner_it_glossary_2018}, including tools and services to
improve asset discovery and management, software security evaluation, hardware
testing, and penetration testing.

The so-called Internet of Things (IoT) brings together a wide variety of
devices, including mobile, wearable, consumer electronics, automotive and
sensors of various types. Such devices use the Internet to connect to other
devices, systems, and applications running on the back-end. Once compromised,
they can be used to attack other devices and systems, steal information, cause
immediate physical damage, or either perform various other malicious actions.
Most of them will likely have long lifespan or less frequent software patches.
The increase of the mesh of devices of diverse technologies brings with it a
considerable increase of the surface of attack. In this scenario, cybersecurity
experts and front-line professionals are now tasked with defeating new types of
attacks that come with increasing frequency.

Intrusion Detection Systems (IDS) are important tools for protecting corporate
networks. From a research point of view, intrusion detection has been a
challenge over the years and most of the related work rely on Data Mining (DM)
or Machine Learning (ML) techniques to detect attacks from known patterns or to
discover new patterns \cite{buczak2016survey,mitchell2014survey}. With
traditional data mining methods, the data set is static and can be traversed
repeatedly, and the detection of new attack patterns requires a new cycle of
training, testing, and dissemination of new models. Unlike traditional methods,
stream mining algorithms can be applied to intrusion detection with several
advantages, such as: ($ i $) working with limited memory, which allows the
implementation in small devices (for example, on the edge from the Web); ($ ii
$) processing traffic data with a single read; ($ iii $) producing real-time
response; and ($ iv $) detecting novelty and changes in concepts already
learned.

%{\color{red} 
Online intrusion detection can be a hard job   depending on the number of devices and their physical location.
With hundreds or thousands of IoT devices and objects scattered across corporate networks or smart cities, moving the traffic data from the devices where they are collected to be analyzed on a traditional cloud or datacenter can be costly or prohibitive due to high latency. Also, moving torrents of traffic data from a large number of devices to be scanned in a centralized infrastructure is not scalable. The current cloud computing paradigm will hardly be able to meet the requirements of low latency and scalability to support intrusion detection \cite{dastjerdi2016fog}.
To face this challenge, it is possible to approximate the intrusion detection function processing to small devices and objects, taking advantage of the resources available on small devices that populate the edge of the network.
%}

\begin{table*}[htb]
	\caption{Summary of related works}
	\centering
	\begin{scriptsize}
		%	\begin{tabularx}{\textwidth}{ l m{2cm} | m{2cm} l m{2cm} | m{2cm} | m{2cm} | m{2cm} }
		\begin{tabularx}{\textwidth}{l|l|X|X|X}	
			%	\begin{tabularx}{\textwidth}{l|l|X|X|m{3.0cm}}
%			Trabalho & Platforma  & Técnica & DataSet & Métricas \\
			Work & Platforms & Technique & DataSet & Metrics\\
			\hline
			\hline
			Kasinathan et al\cite{dos-6lowpan-iot} & 6LoWPAN & Suricata & Real data with metasploit & Accuracy, packets/second\\
			\hline
			Sheikhan and Bostani\cite{Hybrid-ids-arch-iot} & 6LoWPAN & Hybrid - MR OPF & NSL-KDD and simulated attacks & FAR and recall\\
			\hline
			Raza et al\cite{SVELTE} & 6LoWPAN & DAG analysis & No information & Recall, energy and memory consumption\\
			\hline
			Furquim et al\cite{Fault-tolerance-disaster} & WSN & MLP of Weka & Real data & MAE, RMSE, R², R, accuracy, recall, precision, specificity\\
			\hline
			Midi et al\cite{Kalis} & WSN & Independent modules, each with one technique & Trace replay and attack injection & Recall, accuracy, memory and CPU consumption\\
			\hline
			Lloret et al\cite{IoT-arch-smartmeter} & Smart City & Clustering, MLP and statistical models & Real data from meters & Water and energy consumption\\
			\hline
			Diffalah et al\cite{scalable-anomaly-detection-smart-city} & Smart City & LiSA, smoothing function & Real data & Outliers, comparison between simulation and collected data\\
			\hline
			Faisal et al\cite{DS-based-IDS-SmartGrid} & \textit{Smart Grid} & 7 MOA classifiers & KDD99 & Accuracy, Kappa, memory consumption, time, FAR and FNR \\
			\hline
		\end{tabularx}
		\label{tab:summary}
	\end{scriptsize}
\end{table*}


\todo[inline]{refazer este paragrafo: O presente trabalho avança a pesquisa sobre um trabalho anterior [ISCC-2019], apresentando as seguintes contribuições: 
(i) a arquitetura proposta em [ISCC] foi instanciada e validada de forma experimental, 
(ii) avaliamos o impacto da distribuiçõão do processamento sobre a qualidade da detecção de novidades quanto da eficiência do processamento 
(iii) discutimos estratégias de distribuição de fluxos para classificação, incluindo a detecção de novidades foram discutidas
}

In the present work, we propose a distributed intrusion detection system for IoT
scenarios with hundreds or thousands of objects and devices. The main
contributions are two. First, we propose an intrusion detection architecture
that leverages cloud edge capabilities, with the goal of reducing latency and
increasing scalability. In addition, we employ and evaluate three Novelty
Detection (ND) methods to learn emerging patterns of network traffic. Our
proposal combines the use of edge network resources to collect and analyze data
streams and public cloud to process offline data for more accurate operations
such as model improvements.


This article is organized as follows: Section \ref{sec:related} presents the
related works. Section \ref{sec:nd} reviews methods for detecting new features.
Section \ref{sec:arch} presents the architecture proposal of the present work.
Section \ref{sec:exper} presents the architecture validation experiments,
encompassing accuracy and processing performance evaluations. Section
\ref{sec:conclusion} summarizes the main findings and presents possible future
work.

%The advent of Internet of Things (IoT) is growing the count and diversity of
% devices on edge networks, this growth increases network traffic patterns and
% extends opportunities for cyber attacks presenting new challenges for network
% administrators. To address those challenges new Network Intrusion Detection
% Systems (NIDS) and architectures can be explored, especially in Fog Computing
% and Data Stream (DS) areas.

% Data Stream (DS)

% - Desafio, resposta, justificativa.
% - Artigo para setembro ou outubro.
% - Revisão dos valores da avaliação.

% ### Desafios, Respostas e Justificativas

% Desafios de arquitetura e validação:

% - Construção de um protótipo da arquitetura IDSA-IoT:
%   - Kafka (Python): Distribuição e balanceamento pelo cluster kafka, hipótese refutada.
%   - Flink (Java ou Scala): Execução do cluster nos dispositivos de névoa, hipótese refutada.
%   - MPI (C e Python): Execução do cluster nos dispositivos de névoa, hipótese aceita.
% - Reimplementação do algoritmo MINAS com fidelidade:
%   - Duas versões: a descrita e a implementação de referência (em Java).
%   - Resolução: utilizar a descrição, não *seguir* a imp. referência, apenas como ponto de comparação. Exemplos:
%     - Definição de raio `r = f * σ` (fator vezes desvio padrão) para `r = max(distance)` (distância máxima);
%     - Tamanho do buffer de desconhecidos e frequência de execução do passo de detecção de novidade;

\begin{highlight}
Expected results:
A system that embraces and explores the inherent distribution of fog computing
in a IoT scenario opposing regular systems where data streams are collected and
centralized before processing;
Impact assessment of the impact of distributed, regional flow characteristics,
local vs global vs distributed forgetting mechanism and other polices.

IDS characteristics and description of physical scenario.

MINAS characteristics.

Distribution and IDSA-IoT architecture.
\end{highlight}

This paper is structured as follows:
Section \ref{sec:related} presents previous works that addresses related
problems and how they influenced our solution.
Section \ref{sec:implementation} address our proposal, the work done, issues
found during implementation and discusses parameters and configurations options
and how we arrived at our choices.
Section \ref{sec:experiments} shows experiments layouts and results, we
compare serial and distributed implementation's metrics for validation,
we also evaluate communication delay effects on classification metrics and
conclude with the speedup per core and overall maximum stream speed.
Section \ref{sec:conclusion} summarizes the research results and presents our
final conclusions and future works.
