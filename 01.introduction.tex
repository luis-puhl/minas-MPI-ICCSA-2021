\section{Introduction {\color{red} Nao mexer por enquanto}} 


\newcommand{\refminas}{\textit{Ref}\xspace}
\newcommand{\mfog}{\textit{MFOG}\xspace}
\newcommand{\iot}{IoT\xspace}
\newcommand{\nids}{NIDS\xspace}
\newcommand{\ds}{DS\xspace}

The advent of Internet of Things (IoT)
is growing the count and diversity of
devices on edge networks, this growth increases network traffic patterns and
extends opportunities for cyber attacks presenting new challenges for network
administrators.
To address those challenges new Network Intrusion Detection Systems (NIDS) and
architectures can be explored, especially in Fog Computing and Data Stream (DS)
areas.

% Data Stream (DS)

% - Desafio, resposta, justificativa.
% - Artigo para setembro ou outubro.
% - Revisão dos valores da avaliação.

% ### Desafios, Respostas e Justificativas

% Desafios de arquitetura e validação:

% - Construção de um protótipo da arquitetura IDSA-IoT:
%   - Kafka (Python): Distribuição e balanceamento pelo cluster kafka, hipótese refutada.
%   - Flink (Java ou Scala): Execução do cluster nos dispositivos de névoa, hipótese refutada.
%   - MPI (C e Python): Execução do cluster nos dispositivos de névoa, hipótese aceita.
% - Reimplementação do algoritmo MINAS com fidelidade:
%   - Duas versões: a descrita e a implementação de referência (em Java).
%   - Resolução: utilizar a descrição, não *seguir* a imp. referência, apenas como ponto de comparação. Exemplos:
%     - Definição de raio `r = f * σ` (fator vezes desvio padrão) para `r = max(distance)` (distância máxima);
%     - Tamanho do buffer de desconhecidos e frequência de execução do passo de detecção de novidade;

\begin{highlight}
Expected results:
A system that embraces and explores the inherent distribution of fog computing
in a IoT scenario opposing regular systems where data streams are collected and
centralized before processing;
Impact assessment of the impact of distributed, regional flow characteristics,
local vs global vs distributed forgetting mechanism and other polices.

IDS characteristics and description of physical scenario.

MINAS characteristics.

Distribution and IDSA-IoT architecture.
\end{highlight}

This paper is structured as follows:
Section \ref{sec:related} presents previous works that addresses related
problems and how they influenced our solution.
Section \ref{sec:implementation} address our proposal, the work done, issues
found during implementation and discusses parameters and configurations options
and how we arrived at our choices.
Section \ref{sec:experiments} shows experiments layouts and results, we
compare serial and distributed implementation's metrics for validation,
we also evaluate communication delay effects on classification metrics and
conclude with the speedup per core and overall maximum stream speed.
Section \ref{sec:conclusion} summarizes the research results and presents our
final conclusions and future works.
