
% \listoftodos
% \todototoc

\section{Introduction}

\newcommand{\refminas}{\textit{Ref}\xspace}
\newcommand{\mfog}{\textit{MFOG}\xspace}
\newcommand{\iot}{IoT\xspace}
\newcommand{\nids}{NIDS\xspace}
\newcommand{\ds}{DS\xspace}

A survey released by Gartner estimated about 20 billion devices connected to the
Internet by 2020, many of them through the IoT~\cite{gartner_forecast_2017}.
More recently, Statista predicted that the total number of such devices will
reach 38 billion by 2025, and 50 billion by 2030 \cite{statista-iot}.
The Internet of Things (IoT) brings together a wide variety of devices,
including mobile, wearable, consumer electronics, automotive and sensors of
various types.
Such devices can either be accessed by users through the Internet or to connect
to other devices, servers, and applications running on the back-end without any
human intervention or supervision
\cite{Tahsien2020,abane2019,haddadpajouh2019survey}.

Security and privacy is a major concern in the IoT, as devices like smart home
assistants and wearable devices hold personal information about the user’s like
their location, health data, and many other sensitive data
\cite{sengupta2020comprehensive}.
Furthermore, once compromised such devices can also be used to attack other
devices and systems, steal information, cause immediate physical damage, or
either perform various other malicious actions.
As additional concerns, IoT devices likely have long lifespan or less frequent
software patches, and the increase of the mesh of devices of diverse
technologies considerably increase the surface of attack.

A study reveals that nearly 20\% of organizations have experienced at least one
IoT-based attack in the last three years.
The study shows that most organizations have no control over the origin and
nature of software and hardware used by connected devices.
To protect against these threats, IoT's worldwide spending on security is
estimated \$3.1 billion in 2021 \cite{gartner_it_glossary_2018}, including tools
and services to improve asset discovery and management, software security
evaluation, hardware testing, and penetration testing.

Because most IoT devices have limited resources (i.e., battery, processing
power, bandwidth, and memory), configurable and expensive algorithm-based
security techniques can not be applied \cite{Zhou2017}.
Machine Learning (ML) techniques have been studied for years to detect attacks
from known patterns or to discover new attacks at an early stage
\cite{buczak2016survey,mitchell2014survey}.
A recent survey in \cite{Tahsien2020} shows that ML based methods are a
promising alternative which can provide potential security tools for the IoT
devices which make them more reliable and accessible than before.
Despite of the promising use of ML to secure IoT systems, most studies presented
in the literature (e.g., those surveyed in
\cite{buczak2016survey,mitchell2014survey,Tahsien2020}) are limited to
traditional ML methods that use static models of traffic behavior.
% With traditional ML methods, the data set is static and can be traversed
% repeatedly, and the detection of new attack patterns requires a new cycle of
% training, testing, and dissemination of new models.
Most existing ML solutions for network-based intrusion detection cannot maintain
their reliability over time when facing evolving attacks \cite{Viegas2019}.
Unlike traditional methods, stream mining algorithms can be applied to intrusion
detection with several advantages, such as: ($ i $) working with limited memory,
which allows the implementation in small devices (for example, on the edge from
the Web); ($ ii $) processing traffic data with a single read; ($ iii $)
producing real-time response; and ($ iv $) detecting novelty and changes in
concepts already learned.

{\color{red} This paper elaborates on the previous work in the following
aspects. First, we instantiated and validated experimentally the IDSA-IoT
architecture (proposed in \cite{Cassales2019a} ) on a distributed system
composed of small devices with limited resources on the edge of the network. Our
implementation can be fully executed on the edge, thus avoiding any dependency
and high latency of cloud resources. We proposed and implemented dMINAS, a
distributed extension of MINAS \cite{Faria2016minas} (a novelty detection for
data streams). We evaluated through experimental method, how the distribution of
the novelty detection affects the capability to detect changes (novelty) in
traffic patterns, and the impact on the computational efficiency. Finally, some
distribution strategies for the novelty detection are discussed. }

This article is organized as follows:
Section \ref{sec:related} presents the related works.
Section \ref{sec:minas} reviews the original algorithm MINAS.
A distributed extension of the novelty detection algorithm, including its
implementation and evaluation are presented in Section \ref{sec:prop}.
Finally, Section \ref{sec:conclusion} summarizes the main findings and presents
possible future work.

\todo{Parei aqui a redação da Introdução.Deixei grifado em vermelho o
paragrafo que fala das contribuições. Talvez com um pequeno esforço em
especificar uma versão distribuida do MINAS teriamos uma extensão do mesmo. }

% Desafios de arquitetura e validação:

% - Construção de um protótipo da arquitetura IDSA-IoT:
%   - Kafka (Python): Distribuição e balanceamento pelo cluster kafka, hipótese refutada.
%   - Flink (Java ou Scala): Execução do cluster nos dispositivos de névoa, hipótese refutada.
%   - MPI (C e Python): Execução do cluster nos dispositivos de névoa, hipótese aceita.
% - Reimplementação do algoritmo MINAS com fidelidade:
%   - Duas versões: a descrita e a implementação de referência (em Java).
%   - Resolução: utilizar a descrição, não *seguir* a imp. referência, apenas como ponto de comparação. Exemplos:
%     - Definição de raio `r = f * σ` (fator vezes desvio padrão) para `r = max(distance)` (distância máxima);
%     - Tamanho do buffer de desconhecidos e frequência de execução do passo de detecção de novidade;

% \begin{highlight}
% Expected results:
% A system that embraces and explores the inherent distribution of fog computing
% in a IoT scenario opposing regular systems where data streams are collected and
% centralized before processing;
% Impact assessment of the impact of distributed, regional flow characteristics,
% local vs global vs distributed forgetting mechanism and other polices.

% IDS characteristics and description of physical scenario.

% MINAS characteristics.

% Distribution and IDSA-IoT architecture.
% \end{highlight}

This paper is structured as follows:
Section \ref{sec:related} presents previous works that addresses related
problems and how they influenced our solution.
Section \ref{sec:implementation} address our proposal, the work done, issues
found during implementation and discusses parameters and configurations options
and how we arrived at our choices.
Section \ref{sec:experiments} shows experiments layouts and results, we
compare serial and distributed implementation's metrics for validation,
we also evaluate communication delay effects on classification metrics and
conclude with the speedup per core and overall maximum stream speed.
Section \ref{sec:conclusion} summarizes the research results and presents our
final conclusions and future works.
