
% ----------------------------------------------------------------------------------------------------------------------
\section{Conclusion} 
\label{sec:conclusion}

% Conexão entre Intro e Conclusão
% - Novelty Detection in Data Streams já é validado
% - IoT leva a ambientes Edge com nós pequenos
% - Nesse ambiente propomos \mfog
%     - Modelamos o software, implementamos
% - Testamos e observamos tempo e qualidade
%     - Tempo adequado mas sub-linear, pouco efeito na qualidade
% - Melhres distribuiçõpes são possíveis
%     - mais roots (melhor algo de distribuição)
%     - mini batching

Novelty Detection in Data Streams (\nd) can be a useful mechanism for Network
Intrusion Detection (\nids) in IoT environments. It can also serve for other related application of \nd using continuous
network or system behavior monitoring and analysis.
% However, in the \iot context, it is expected that small edge devices perform
such maintenance tasks.
Regarding the tremendous amount of data that must be processed in the flow analysis for \nd, it 
is relevant that this processing takes place at the edge of the network. 
However, one relevant shortcoming of the IoT in this case is the reduced processing capacity of such devices. 

In this sense, we have put together and evaluated a distributed architecture for \nd at the network edge.

In that small computing on edge scenario, we propose \mfog: a distributed \nd
implementation based on the \nd algorithm \minas, and, evaluated with a \nids
task with appropriate dataset.
The main goal this work  to observe the effects of our approach to a
previously serial only algorithm, specially in regards to time and quality
metrics.

% \mfog is a demonstration piece

% pode ser a conclusão do abstract
While there is some impact on the predictive metrics this is not reflected on
overall classification quality metrics indicating that distribution of \minas
shows a negligible loss of accuracy.
% but is overshadow by the benefits of scalability.
In regards of time and scale, our distributed executions used less time than
previous implementation but efficient distribution was not achieved as the
observed time as we added nodes remained constant.
% In our experiments, a months worth of traffic flow descriptors is processed in
% around $300$ seconds even with non-optimal load sharing round-robin strategy.
% (single sample round-robin that incurs the bigger networking overhead),
% sparing use of memory (unknown buffer limited to a fraction of the available memory),
% and potential system slowdown due to cascading effects on networking queue
% (caused by the Detector Module when its unknown buffer is full and novelty
% detection task is under way as it still a non-stream oriented algorithm limited
% to a single batch of fixed memory).

Our treatment involved reworking the algorithm and implementation to be
distributed and to minimize the memory usage as to fit in smaller devices.
Other algorithms still need a similar treatment and, more importantly, other
distribution strategies should be considered.

% Aplicação em cenário real.

% ----------------------------------------------------------------------------------------------------------------------
\section*{Acknowledgment}

% The preferred spelling of the word ``acknowledgment'' in America is without an ``e'' after the ``g''.
% Avoid the stilted expression ``one of us (R. B. G.) thanks $\ldots$''.
% Instead, try ``R. B. G. thanks$\ldots$''.
% Put sponsor  acknowledgments in the unnumbered footnote on the first page.
The authors thank CNPq (Contract 167345/2018-4).
Hermes Senger also thanks CNPq (Contract 305032/2015-1) and FAPESP (Contract
2018/00452-2, and Contract 2015/24461-2) for their support.
