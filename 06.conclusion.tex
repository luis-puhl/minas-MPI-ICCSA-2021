
% ----------------------------------------------------------------------------------------------------------------------
\section{Conclusion} 
\label{sec:conclusion}

% Conexão entre Intro e Conclusão
% - Novelty Detection in Data Streams já é validado
% - IoT leva a ambientes Edge com nós pequenos
% - Nesse ambiente propomos \mfog
%     - Modelamos o software, implementamos
% - Testamos e observamos tempo e qualidade
%     - Tempo adequado mas sub-linear, pouco efeito na qualidade
% - Melhres distribuiçõpes são possíveis
%     - mais roots (melhor algo de distribuição)
%     - mini batching

Data Stream Novelty Detection (\nd) can be a useful mechanism for Network
Intrusion Detection (\nids) in IoT environments. It can also serve other related applications of \nd using continuous
network or system behavior monitoring and analysis.
% However, in the \iot context, it is expected that small edge devices perform
% such maintenance tasks.
Regarding the tremendous amount of data that must be processed in the flow analysis for \nd, it 
is relevant that this processing takes place at the edge of the network. 
However, one relevant shortcoming of the IoT, in this case, is the reduced processing capacity of such edge devices. 

In this sense, we have put together and evaluated a distributed architecture for
performing \nd applied at network flow descriptors at the edge.
% In that small computing on edge scenario, w
Our proposal, \mfog, is a distributed \nd implementation based on the \minas
algorithm and the main goal of this work is to observe the effects of our
approach to a previously sequential only algorithm, especially in regards to
time and quality metrics.

% \mfog is a demonstration piece

% pode ser a conclusão do abstract
While there is some impact on the predictive metrics, this is not reflected on
overall classification quality metrics indicating that distribution of \minas
has a negligible loss of accuracy.
% but is overshadow by the benefits of scalability.
In regards to time and scale, our distributed executions was faster than the 
previous sequential implementation of \minas, but efficient data distribution was not achieved as the
observed time with each added node remained near constant.
% In our experiments, a months worth of traffic flow descriptors is processed in
% around $300$ seconds even with non-optimal load sharing round-robin strategy.
% (single sample round-robin that incurs the bigger networking overhead),
% sparing use of memory (unknown buffer limited to a fraction of the available memory),
% and potential system slowdown due to cascading effects on networking queue
% (caused by the Detector Module when its unknown buffer is full and novelty
% detection task is under way as it still a non-stream oriented algorithm limited
% to a single batch of fixed memory).

% Acho que trocaria a úlitma frase por algo mais conclusivo para fechar o artigo
Overall, \mfog and the idea of using distributed flow classification and novelty
detection while minimizing memory usage to fit in smaller devices at the edge of
the network is a viable and promising solution.
Further work includes the investigation of other \nd algorithms, other clustering
algorithms in \minas and analysis of varying load balancing strategies.
% also considering using a distributed version of the k-means clustering.

% Our treatment involved reworking the algorithm and implementation to be
% distributed and %to minimize the memory usage as 
% minimizing memory usage to fit in smaller devices.
% Other algorithms still need a similar treatment and, more importantly, other
% distribution strategies should be considered.

% Aplicação em cenário real.

% ----------------------------------------------------------------------------------------------------------------------
\section*{Acknowledgment}

% The preferred spelling of the word ``acknowledgment'' in America is without an ``e'' after the ``g''.
% Avoid the stilted expression ``one of us (R. B. G.) thanks $\ldots$''.
% Instead, try ``R. B. G. thanks$\ldots$''.
% Put sponsor  acknowledgments in the unnumbered footnote on the first page.
This study was financed in part by the Coordenação de Aperfeiçoamento de Pessoal de
Nível Superior - Brasil (CAPES) - Finance Code 001,
and Programa Institucional de Internacionalização – CAPES-PrInt UFSCar (Contract 88887.373234/2019-00). 
Authors also thank Stic AMSUD (project 20-STIC-09), FAPESP (contract numbers  2018/22979-2, and 2015/24461-2) and CNPq (Contract 167345/2018-4) for their support.
