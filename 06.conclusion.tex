
% ----------------------------------------------------------------------------------------------------------------------
\section{Conclusion} 
\label{sec:conclusion}

% Conexão entre Intro e Conclusão
% - Novelty Detection in Data Streams já é validado
% - IoT leva a ambientes Edge com nós pequenos
% - Nesse ambiente propomos \mfog
%     - Modelamos o software, implementamos
% - Testamos e observamos tempo e qualidade
%     - Tempo adequado mas sub-linear, pouco efeito na qualidade
% - Melhres distribuiçõpes são possíveis
%     - mais roots (melhor algo de distribuição)
%     - mini batching

Novelty Detection (ND) in Data Streams (DS) is a useful tool for
IoT Network Intrusion Detection (\nids) (or other related application of ND in DS),
in this context is expected that small edge devices perform such maintenance tasks.
In that small edge computing ambient, we propose \mfog: a distributed ND implementation
based on the ND algorithm \minas, and, evaluated with a \nids task with appropriate dataset.
The main goal of our proposal is to observe the effects on the results of such approach to
a previously serial only algorithm, specially in regards to time and quality metrics.

% \mfog is a demonstration piece

% pode ser a conclusão do abstract
While there is some impact on the predictive metrics this is not reflected on
classification quality metrics indicating that distribution of \minas
resulted some penalty but not a significant.
% but is overshadow by the benefits of scalability.
Beside that, in our experiments, a months worth of traffic flow descriptors is
processed in around $300$ seconds even with non-optimal load sharing round-robin strategy.
% (single sample round-robin that incurs the bigger networking overhead),
% sparing use of memory (unknown buffer limited to a fraction of the available memory),
% and potential system slowdown due to cascading effects on networking queue
% (caused by the Detector Module when its unknown buffer is full and novelty
% detection task is under way as it still a non-stream oriented algorithm limited
% to a single batch of fixed memory).

Our treatment involved reworking the algorithm and implementation to be
distributed and to minimize the memory usage as to fit in smaller devices.
Other algorithms still need a similar treatment and, more importantly, other
distribution strategies should be considered.

% Aplicação em cenário real.

% ----------------------------------------------------------------------------------------------------------------------
\section*{Acknowledgment}

% The preferred spelling of the word ``acknowledgment'' in America is without an ``e'' after the ``g''.
% Avoid the stilted expression ``one of us (R. B. G.) thanks $\ldots$''.
% Instead, try ``R. B. G. thanks$\ldots$''.
% Put sponsor  acknowledgments in the unnumbered footnote on the first page.
The authors thank CNPq (Contract 167345/2018-4).
Hermes Senger also thanks CNPq (Contract 305032/2015-1) and FAPESP (Contract
2018/00452-2, and Contract 2015/24461-2) for their support.
